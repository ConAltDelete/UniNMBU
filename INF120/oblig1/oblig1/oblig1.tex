
    




    
\documentclass[11pt]{article}

    
    \usepackage[breakable]{tcolorbox}
    \tcbset{nobeforeafter} % prevents tcolorboxes being placing in paragraphs
    \usepackage{float}
    \floatplacement{figure}{H} % forces figures to be placed at the correct location
    
    \usepackage[T1]{fontenc}
    % Nicer default font (+ math font) than Computer Modern for most use cases
    \usepackage{mathpazo}

    % Basic figure setup, for now with no caption control since it's done
    % automatically by Pandoc (which extracts ![](path) syntax from Markdown).
    \usepackage{graphicx}
    % We will generate all images so they have a width \maxwidth. This means
    % that they will get their normal width if they fit onto the page, but
    % are scaled down if they would overflow the margins.
    \makeatletter
    \def\maxwidth{\ifdim\Gin@nat@width>\linewidth\linewidth
    \else\Gin@nat@width\fi}
    \makeatother
    \let\Oldincludegraphics\includegraphics
    % Set max figure width to be 80% of text width, for now hardcoded.
    \renewcommand{\includegraphics}[1]{\Oldincludegraphics[width=.8\maxwidth]{#1}}
    % Ensure that by default, figures have no caption (until we provide a
    % proper Figure object with a Caption API and a way to capture that
    % in the conversion process - todo).
    \usepackage{caption}
    \DeclareCaptionLabelFormat{nolabel}{}
    \captionsetup{labelformat=nolabel}

    \usepackage{adjustbox} % Used to constrain images to a maximum size 
    \usepackage{xcolor} % Allow colors to be defined
    \usepackage{enumerate} % Needed for markdown enumerations to work
    \usepackage{geometry} % Used to adjust the document margins
    \usepackage{amsmath} % Equations
    \usepackage{amssymb} % Equations
    \usepackage{textcomp} % defines textquotesingle
    % Hack from http://tex.stackexchange.com/a/47451/13684:
    \AtBeginDocument{%
        \def\PYZsq{\textquotesingle}% Upright quotes in Pygmentized code
    }
    \usepackage{upquote} % Upright quotes for verbatim code
    \usepackage{eurosym} % defines \euro
    \usepackage[mathletters]{ucs} % Extended unicode (utf-8) support
    \usepackage[utf8x]{inputenc} % Allow utf-8 characters in the tex document
    \usepackage{fancyvrb} % verbatim replacement that allows latex
    \usepackage{grffile} % extends the file name processing of package graphics 
                         % to support a larger range 
    % The hyperref package gives us a pdf with properly built
    % internal navigation ('pdf bookmarks' for the table of contents,
    % internal cross-reference links, web links for URLs, etc.)
    \usepackage{hyperref}
    \usepackage{longtable} % longtable support required by pandoc >1.10
    \usepackage{booktabs}  % table support for pandoc > 1.12.2
    \usepackage[inline]{enumitem} % IRkernel/repr support (it uses the enumerate* environment)
    \usepackage[normalem]{ulem} % ulem is needed to support strikethroughs (\sout)
                                % normalem makes italics be italics, not underlines
    \usepackage{mathrsfs}
    

    
    % Colors for the hyperref package
    \definecolor{urlcolor}{rgb}{0,.145,.698}
    \definecolor{linkcolor}{rgb}{.71,0.21,0.01}
    \definecolor{citecolor}{rgb}{.12,.54,.11}

    % ANSI colors
    \definecolor{ansi-black}{HTML}{3E424D}
    \definecolor{ansi-black-intense}{HTML}{282C36}
    \definecolor{ansi-red}{HTML}{E75C58}
    \definecolor{ansi-red-intense}{HTML}{B22B31}
    \definecolor{ansi-green}{HTML}{00A250}
    \definecolor{ansi-green-intense}{HTML}{007427}
    \definecolor{ansi-yellow}{HTML}{DDB62B}
    \definecolor{ansi-yellow-intense}{HTML}{B27D12}
    \definecolor{ansi-blue}{HTML}{208FFB}
    \definecolor{ansi-blue-intense}{HTML}{0065CA}
    \definecolor{ansi-magenta}{HTML}{D160C4}
    \definecolor{ansi-magenta-intense}{HTML}{A03196}
    \definecolor{ansi-cyan}{HTML}{60C6C8}
    \definecolor{ansi-cyan-intense}{HTML}{258F8F}
    \definecolor{ansi-white}{HTML}{C5C1B4}
    \definecolor{ansi-white-intense}{HTML}{A1A6B2}
    \definecolor{ansi-default-inverse-fg}{HTML}{FFFFFF}
    \definecolor{ansi-default-inverse-bg}{HTML}{000000}

    % commands and environments needed by pandoc snippets
    % extracted from the output of `pandoc -s`
    \providecommand{\tightlist}{%
      \setlength{\itemsep}{0pt}\setlength{\parskip}{0pt}}
    \DefineVerbatimEnvironment{Highlighting}{Verbatim}{commandchars=\\\{\}}
    % Add ',fontsize=\small' for more characters per line
    \newenvironment{Shaded}{}{}
    \newcommand{\KeywordTok}[1]{\textcolor[rgb]{0.00,0.44,0.13}{\textbf{{#1}}}}
    \newcommand{\DataTypeTok}[1]{\textcolor[rgb]{0.56,0.13,0.00}{{#1}}}
    \newcommand{\DecValTok}[1]{\textcolor[rgb]{0.25,0.63,0.44}{{#1}}}
    \newcommand{\BaseNTok}[1]{\textcolor[rgb]{0.25,0.63,0.44}{{#1}}}
    \newcommand{\FloatTok}[1]{\textcolor[rgb]{0.25,0.63,0.44}{{#1}}}
    \newcommand{\CharTok}[1]{\textcolor[rgb]{0.25,0.44,0.63}{{#1}}}
    \newcommand{\StringTok}[1]{\textcolor[rgb]{0.25,0.44,0.63}{{#1}}}
    \newcommand{\CommentTok}[1]{\textcolor[rgb]{0.38,0.63,0.69}{\textit{{#1}}}}
    \newcommand{\OtherTok}[1]{\textcolor[rgb]{0.00,0.44,0.13}{{#1}}}
    \newcommand{\AlertTok}[1]{\textcolor[rgb]{1.00,0.00,0.00}{\textbf{{#1}}}}
    \newcommand{\FunctionTok}[1]{\textcolor[rgb]{0.02,0.16,0.49}{{#1}}}
    \newcommand{\RegionMarkerTok}[1]{{#1}}
    \newcommand{\ErrorTok}[1]{\textcolor[rgb]{1.00,0.00,0.00}{\textbf{{#1}}}}
    \newcommand{\NormalTok}[1]{{#1}}
    
    % Additional commands for more recent versions of Pandoc
    \newcommand{\ConstantTok}[1]{\textcolor[rgb]{0.53,0.00,0.00}{{#1}}}
    \newcommand{\SpecialCharTok}[1]{\textcolor[rgb]{0.25,0.44,0.63}{{#1}}}
    \newcommand{\VerbatimStringTok}[1]{\textcolor[rgb]{0.25,0.44,0.63}{{#1}}}
    \newcommand{\SpecialStringTok}[1]{\textcolor[rgb]{0.73,0.40,0.53}{{#1}}}
    \newcommand{\ImportTok}[1]{{#1}}
    \newcommand{\DocumentationTok}[1]{\textcolor[rgb]{0.73,0.13,0.13}{\textit{{#1}}}}
    \newcommand{\AnnotationTok}[1]{\textcolor[rgb]{0.38,0.63,0.69}{\textbf{\textit{{#1}}}}}
    \newcommand{\CommentVarTok}[1]{\textcolor[rgb]{0.38,0.63,0.69}{\textbf{\textit{{#1}}}}}
    \newcommand{\VariableTok}[1]{\textcolor[rgb]{0.10,0.09,0.49}{{#1}}}
    \newcommand{\ControlFlowTok}[1]{\textcolor[rgb]{0.00,0.44,0.13}{\textbf{{#1}}}}
    \newcommand{\OperatorTok}[1]{\textcolor[rgb]{0.40,0.40,0.40}{{#1}}}
    \newcommand{\BuiltInTok}[1]{{#1}}
    \newcommand{\ExtensionTok}[1]{{#1}}
    \newcommand{\PreprocessorTok}[1]{\textcolor[rgb]{0.74,0.48,0.00}{{#1}}}
    \newcommand{\AttributeTok}[1]{\textcolor[rgb]{0.49,0.56,0.16}{{#1}}}
    \newcommand{\InformationTok}[1]{\textcolor[rgb]{0.38,0.63,0.69}{\textbf{\textit{{#1}}}}}
    \newcommand{\WarningTok}[1]{\textcolor[rgb]{0.38,0.63,0.69}{\textbf{\textit{{#1}}}}}
    
    
    % Define a nice break command that doesn't care if a line doesn't already
    % exist.
    \def\br{\hspace*{\fill} \\* }
    % Math Jax compatibility definitions
    \def\gt{>}
    \def\lt{<}
    \let\Oldtex\TeX
    \let\Oldlatex\LaTeX
    \renewcommand{\TeX}{\textrm{\Oldtex}}
    \renewcommand{\LaTeX}{\textrm{\Oldlatex}}
    % Document parameters
    % Document title
    \title{Oblig 1 i INF120}
    
    
    
    
    
% Pygments definitions
\makeatletter
\def\PY@reset{\let\PY@it=\relax \let\PY@bf=\relax%
    \let\PY@ul=\relax \let\PY@tc=\relax%
    \let\PY@bc=\relax \let\PY@ff=\relax}
\def\PY@tok#1{\csname PY@tok@#1\endcsname}
\def\PY@toks#1+{\ifx\relax#1\empty\else%
    \PY@tok{#1}\expandafter\PY@toks\fi}
\def\PY@do#1{\PY@bc{\PY@tc{\PY@ul{%
    \PY@it{\PY@bf{\PY@ff{#1}}}}}}}
\def\PY#1#2{\PY@reset\PY@toks#1+\relax+\PY@do{#2}}

\expandafter\def\csname PY@tok@w\endcsname{\def\PY@tc##1{\textcolor[rgb]{0.73,0.73,0.73}{##1}}}
\expandafter\def\csname PY@tok@c\endcsname{\let\PY@it=\textit\def\PY@tc##1{\textcolor[rgb]{0.25,0.50,0.50}{##1}}}
\expandafter\def\csname PY@tok@cp\endcsname{\def\PY@tc##1{\textcolor[rgb]{0.74,0.48,0.00}{##1}}}
\expandafter\def\csname PY@tok@k\endcsname{\let\PY@bf=\textbf\def\PY@tc##1{\textcolor[rgb]{0.00,0.50,0.00}{##1}}}
\expandafter\def\csname PY@tok@kp\endcsname{\def\PY@tc##1{\textcolor[rgb]{0.00,0.50,0.00}{##1}}}
\expandafter\def\csname PY@tok@kt\endcsname{\def\PY@tc##1{\textcolor[rgb]{0.69,0.00,0.25}{##1}}}
\expandafter\def\csname PY@tok@o\endcsname{\def\PY@tc##1{\textcolor[rgb]{0.40,0.40,0.40}{##1}}}
\expandafter\def\csname PY@tok@ow\endcsname{\let\PY@bf=\textbf\def\PY@tc##1{\textcolor[rgb]{0.67,0.13,1.00}{##1}}}
\expandafter\def\csname PY@tok@nb\endcsname{\def\PY@tc##1{\textcolor[rgb]{0.00,0.50,0.00}{##1}}}
\expandafter\def\csname PY@tok@nf\endcsname{\def\PY@tc##1{\textcolor[rgb]{0.00,0.00,1.00}{##1}}}
\expandafter\def\csname PY@tok@nc\endcsname{\let\PY@bf=\textbf\def\PY@tc##1{\textcolor[rgb]{0.00,0.00,1.00}{##1}}}
\expandafter\def\csname PY@tok@nn\endcsname{\let\PY@bf=\textbf\def\PY@tc##1{\textcolor[rgb]{0.00,0.00,1.00}{##1}}}
\expandafter\def\csname PY@tok@ne\endcsname{\let\PY@bf=\textbf\def\PY@tc##1{\textcolor[rgb]{0.82,0.25,0.23}{##1}}}
\expandafter\def\csname PY@tok@nv\endcsname{\def\PY@tc##1{\textcolor[rgb]{0.10,0.09,0.49}{##1}}}
\expandafter\def\csname PY@tok@no\endcsname{\def\PY@tc##1{\textcolor[rgb]{0.53,0.00,0.00}{##1}}}
\expandafter\def\csname PY@tok@nl\endcsname{\def\PY@tc##1{\textcolor[rgb]{0.63,0.63,0.00}{##1}}}
\expandafter\def\csname PY@tok@ni\endcsname{\let\PY@bf=\textbf\def\PY@tc##1{\textcolor[rgb]{0.60,0.60,0.60}{##1}}}
\expandafter\def\csname PY@tok@na\endcsname{\def\PY@tc##1{\textcolor[rgb]{0.49,0.56,0.16}{##1}}}
\expandafter\def\csname PY@tok@nt\endcsname{\let\PY@bf=\textbf\def\PY@tc##1{\textcolor[rgb]{0.00,0.50,0.00}{##1}}}
\expandafter\def\csname PY@tok@nd\endcsname{\def\PY@tc##1{\textcolor[rgb]{0.67,0.13,1.00}{##1}}}
\expandafter\def\csname PY@tok@s\endcsname{\def\PY@tc##1{\textcolor[rgb]{0.73,0.13,0.13}{##1}}}
\expandafter\def\csname PY@tok@sd\endcsname{\let\PY@it=\textit\def\PY@tc##1{\textcolor[rgb]{0.73,0.13,0.13}{##1}}}
\expandafter\def\csname PY@tok@si\endcsname{\let\PY@bf=\textbf\def\PY@tc##1{\textcolor[rgb]{0.73,0.40,0.53}{##1}}}
\expandafter\def\csname PY@tok@se\endcsname{\let\PY@bf=\textbf\def\PY@tc##1{\textcolor[rgb]{0.73,0.40,0.13}{##1}}}
\expandafter\def\csname PY@tok@sr\endcsname{\def\PY@tc##1{\textcolor[rgb]{0.73,0.40,0.53}{##1}}}
\expandafter\def\csname PY@tok@ss\endcsname{\def\PY@tc##1{\textcolor[rgb]{0.10,0.09,0.49}{##1}}}
\expandafter\def\csname PY@tok@sx\endcsname{\def\PY@tc##1{\textcolor[rgb]{0.00,0.50,0.00}{##1}}}
\expandafter\def\csname PY@tok@m\endcsname{\def\PY@tc##1{\textcolor[rgb]{0.40,0.40,0.40}{##1}}}
\expandafter\def\csname PY@tok@gh\endcsname{\let\PY@bf=\textbf\def\PY@tc##1{\textcolor[rgb]{0.00,0.00,0.50}{##1}}}
\expandafter\def\csname PY@tok@gu\endcsname{\let\PY@bf=\textbf\def\PY@tc##1{\textcolor[rgb]{0.50,0.00,0.50}{##1}}}
\expandafter\def\csname PY@tok@gd\endcsname{\def\PY@tc##1{\textcolor[rgb]{0.63,0.00,0.00}{##1}}}
\expandafter\def\csname PY@tok@gi\endcsname{\def\PY@tc##1{\textcolor[rgb]{0.00,0.63,0.00}{##1}}}
\expandafter\def\csname PY@tok@gr\endcsname{\def\PY@tc##1{\textcolor[rgb]{1.00,0.00,0.00}{##1}}}
\expandafter\def\csname PY@tok@ge\endcsname{\let\PY@it=\textit}
\expandafter\def\csname PY@tok@gs\endcsname{\let\PY@bf=\textbf}
\expandafter\def\csname PY@tok@gp\endcsname{\let\PY@bf=\textbf\def\PY@tc##1{\textcolor[rgb]{0.00,0.00,0.50}{##1}}}
\expandafter\def\csname PY@tok@go\endcsname{\def\PY@tc##1{\textcolor[rgb]{0.53,0.53,0.53}{##1}}}
\expandafter\def\csname PY@tok@gt\endcsname{\def\PY@tc##1{\textcolor[rgb]{0.00,0.27,0.87}{##1}}}
\expandafter\def\csname PY@tok@err\endcsname{\def\PY@bc##1{\setlength{\fboxsep}{0pt}\fcolorbox[rgb]{1.00,0.00,0.00}{1,1,1}{\strut ##1}}}
\expandafter\def\csname PY@tok@kc\endcsname{\let\PY@bf=\textbf\def\PY@tc##1{\textcolor[rgb]{0.00,0.50,0.00}{##1}}}
\expandafter\def\csname PY@tok@kd\endcsname{\let\PY@bf=\textbf\def\PY@tc##1{\textcolor[rgb]{0.00,0.50,0.00}{##1}}}
\expandafter\def\csname PY@tok@kn\endcsname{\let\PY@bf=\textbf\def\PY@tc##1{\textcolor[rgb]{0.00,0.50,0.00}{##1}}}
\expandafter\def\csname PY@tok@kr\endcsname{\let\PY@bf=\textbf\def\PY@tc##1{\textcolor[rgb]{0.00,0.50,0.00}{##1}}}
\expandafter\def\csname PY@tok@bp\endcsname{\def\PY@tc##1{\textcolor[rgb]{0.00,0.50,0.00}{##1}}}
\expandafter\def\csname PY@tok@fm\endcsname{\def\PY@tc##1{\textcolor[rgb]{0.00,0.00,1.00}{##1}}}
\expandafter\def\csname PY@tok@vc\endcsname{\def\PY@tc##1{\textcolor[rgb]{0.10,0.09,0.49}{##1}}}
\expandafter\def\csname PY@tok@vg\endcsname{\def\PY@tc##1{\textcolor[rgb]{0.10,0.09,0.49}{##1}}}
\expandafter\def\csname PY@tok@vi\endcsname{\def\PY@tc##1{\textcolor[rgb]{0.10,0.09,0.49}{##1}}}
\expandafter\def\csname PY@tok@vm\endcsname{\def\PY@tc##1{\textcolor[rgb]{0.10,0.09,0.49}{##1}}}
\expandafter\def\csname PY@tok@sa\endcsname{\def\PY@tc##1{\textcolor[rgb]{0.73,0.13,0.13}{##1}}}
\expandafter\def\csname PY@tok@sb\endcsname{\def\PY@tc##1{\textcolor[rgb]{0.73,0.13,0.13}{##1}}}
\expandafter\def\csname PY@tok@sc\endcsname{\def\PY@tc##1{\textcolor[rgb]{0.73,0.13,0.13}{##1}}}
\expandafter\def\csname PY@tok@dl\endcsname{\def\PY@tc##1{\textcolor[rgb]{0.73,0.13,0.13}{##1}}}
\expandafter\def\csname PY@tok@s2\endcsname{\def\PY@tc##1{\textcolor[rgb]{0.73,0.13,0.13}{##1}}}
\expandafter\def\csname PY@tok@sh\endcsname{\def\PY@tc##1{\textcolor[rgb]{0.73,0.13,0.13}{##1}}}
\expandafter\def\csname PY@tok@s1\endcsname{\def\PY@tc##1{\textcolor[rgb]{0.73,0.13,0.13}{##1}}}
\expandafter\def\csname PY@tok@mb\endcsname{\def\PY@tc##1{\textcolor[rgb]{0.40,0.40,0.40}{##1}}}
\expandafter\def\csname PY@tok@mf\endcsname{\def\PY@tc##1{\textcolor[rgb]{0.40,0.40,0.40}{##1}}}
\expandafter\def\csname PY@tok@mh\endcsname{\def\PY@tc##1{\textcolor[rgb]{0.40,0.40,0.40}{##1}}}
\expandafter\def\csname PY@tok@mi\endcsname{\def\PY@tc##1{\textcolor[rgb]{0.40,0.40,0.40}{##1}}}
\expandafter\def\csname PY@tok@il\endcsname{\def\PY@tc##1{\textcolor[rgb]{0.40,0.40,0.40}{##1}}}
\expandafter\def\csname PY@tok@mo\endcsname{\def\PY@tc##1{\textcolor[rgb]{0.40,0.40,0.40}{##1}}}
\expandafter\def\csname PY@tok@ch\endcsname{\let\PY@it=\textit\def\PY@tc##1{\textcolor[rgb]{0.25,0.50,0.50}{##1}}}
\expandafter\def\csname PY@tok@cm\endcsname{\let\PY@it=\textit\def\PY@tc##1{\textcolor[rgb]{0.25,0.50,0.50}{##1}}}
\expandafter\def\csname PY@tok@cpf\endcsname{\let\PY@it=\textit\def\PY@tc##1{\textcolor[rgb]{0.25,0.50,0.50}{##1}}}
\expandafter\def\csname PY@tok@c1\endcsname{\let\PY@it=\textit\def\PY@tc##1{\textcolor[rgb]{0.25,0.50,0.50}{##1}}}
\expandafter\def\csname PY@tok@cs\endcsname{\let\PY@it=\textit\def\PY@tc##1{\textcolor[rgb]{0.25,0.50,0.50}{##1}}}

\def\PYZbs{\char`\\}
\def\PYZus{\char`\_}
\def\PYZob{\char`\{}
\def\PYZcb{\char`\}}
\def\PYZca{\char`\^}
\def\PYZam{\char`\&}
\def\PYZlt{\char`\<}
\def\PYZgt{\char`\>}
\def\PYZsh{\char`\#}
\def\PYZpc{\char`\%}
\def\PYZdl{\char`\$}
\def\PYZhy{\char`\-}
\def\PYZsq{\char`\'}
\def\PYZdq{\char`\"}
\def\PYZti{\char`\~}
% for compatibility with earlier versions
\def\PYZat{@}
\def\PYZlb{[}
\def\PYZrb{]}
\makeatother


    % For linebreaks inside Verbatim environment from package fancyvrb. 
    \makeatletter
        \newbox\Wrappedcontinuationbox 
        \newbox\Wrappedvisiblespacebox 
        \newcommand*\Wrappedvisiblespace {\textcolor{red}{\textvisiblespace}} 
        \newcommand*\Wrappedcontinuationsymbol {\textcolor{red}{\llap{\tiny$\m@th\hookrightarrow$}}} 
        \newcommand*\Wrappedcontinuationindent {3ex } 
        \newcommand*\Wrappedafterbreak {\kern\Wrappedcontinuationindent\copy\Wrappedcontinuationbox} 
        % Take advantage of the already applied Pygments mark-up to insert 
        % potential linebreaks for TeX processing. 
        %        {, <, #, %, $, ' and ": go to next line. 
        %        _, }, ^, &, >, - and ~: stay at end of broken line. 
        % Use of \textquotesingle for straight quote. 
        \newcommand*\Wrappedbreaksatspecials {% 
            \def\PYGZus{\discretionary{\char`\_}{\Wrappedafterbreak}{\char`\_}}% 
            \def\PYGZob{\discretionary{}{\Wrappedafterbreak\char`\{}{\char`\{}}% 
            \def\PYGZcb{\discretionary{\char`\}}{\Wrappedafterbreak}{\char`\}}}% 
            \def\PYGZca{\discretionary{\char`\^}{\Wrappedafterbreak}{\char`\^}}% 
            \def\PYGZam{\discretionary{\char`\&}{\Wrappedafterbreak}{\char`\&}}% 
            \def\PYGZlt{\discretionary{}{\Wrappedafterbreak\char`\<}{\char`\<}}% 
            \def\PYGZgt{\discretionary{\char`\>}{\Wrappedafterbreak}{\char`\>}}% 
            \def\PYGZsh{\discretionary{}{\Wrappedafterbreak\char`\#}{\char`\#}}% 
            \def\PYGZpc{\discretionary{}{\Wrappedafterbreak\char`\%}{\char`\%}}% 
            \def\PYGZdl{\discretionary{}{\Wrappedafterbreak\char`\$}{\char`\$}}% 
            \def\PYGZhy{\discretionary{\char`\-}{\Wrappedafterbreak}{\char`\-}}% 
            \def\PYGZsq{\discretionary{}{\Wrappedafterbreak\textquotesingle}{\textquotesingle}}% 
            \def\PYGZdq{\discretionary{}{\Wrappedafterbreak\char`\"}{\char`\"}}% 
            \def\PYGZti{\discretionary{\char`\~}{\Wrappedafterbreak}{\char`\~}}% 
        } 
        % Some characters . , ; ? ! / are not pygmentized. 
        % This macro makes them "active" and they will insert potential linebreaks 
        \newcommand*\Wrappedbreaksatpunct {% 
            \lccode`\~`\.\lowercase{\def~}{\discretionary{\hbox{\char`\.}}{\Wrappedafterbreak}{\hbox{\char`\.}}}% 
            \lccode`\~`\,\lowercase{\def~}{\discretionary{\hbox{\char`\,}}{\Wrappedafterbreak}{\hbox{\char`\,}}}% 
            \lccode`\~`\;\lowercase{\def~}{\discretionary{\hbox{\char`\;}}{\Wrappedafterbreak}{\hbox{\char`\;}}}% 
            \lccode`\~`\:\lowercase{\def~}{\discretionary{\hbox{\char`\:}}{\Wrappedafterbreak}{\hbox{\char`\:}}}% 
            \lccode`\~`\?\lowercase{\def~}{\discretionary{\hbox{\char`\?}}{\Wrappedafterbreak}{\hbox{\char`\?}}}% 
            \lccode`\~`\!\lowercase{\def~}{\discretionary{\hbox{\char`\!}}{\Wrappedafterbreak}{\hbox{\char`\!}}}% 
            \lccode`\~`\/\lowercase{\def~}{\discretionary{\hbox{\char`\/}}{\Wrappedafterbreak}{\hbox{\char`\/}}}% 
            \catcode`\.\active
            \catcode`\,\active 
            \catcode`\;\active
            \catcode`\:\active
            \catcode`\?\active
            \catcode`\!\active
            \catcode`\/\active 
            \lccode`\~`\~ 	
        }
    \makeatother

    \let\OriginalVerbatim=\Verbatim
    \makeatletter
    \renewcommand{\Verbatim}[1][1]{%
        %\parskip\z@skip
        \sbox\Wrappedcontinuationbox {\Wrappedcontinuationsymbol}%
        \sbox\Wrappedvisiblespacebox {\FV@SetupFont\Wrappedvisiblespace}%
        \def\FancyVerbFormatLine ##1{\hsize\linewidth
            \vtop{\raggedright\hyphenpenalty\z@\exhyphenpenalty\z@
                \doublehyphendemerits\z@\finalhyphendemerits\z@
                \strut ##1\strut}%
        }%
        % If the linebreak is at a space, the latter will be displayed as visible
        % space at end of first line, and a continuation symbol starts next line.
        % Stretch/shrink are however usually zero for typewriter font.
        \def\FV@Space {%
            \nobreak\hskip\z@ plus\fontdimen3\font minus\fontdimen4\font
            \discretionary{\copy\Wrappedvisiblespacebox}{\Wrappedafterbreak}
            {\kern\fontdimen2\font}%
        }%
        
        % Allow breaks at special characters using \PYG... macros.
        \Wrappedbreaksatspecials
        % Breaks at punctuation characters . , ; ? ! and / need catcode=\active 	
        \OriginalVerbatim[#1,codes*=\Wrappedbreaksatpunct]%
    }
    \makeatother

    % Exact colors from NB
    \definecolor{incolor}{HTML}{303F9F}
    \definecolor{outcolor}{HTML}{D84315}
    \definecolor{cellborder}{HTML}{CFCFCF}
    \definecolor{cellbackground}{HTML}{F7F7F7}
    
    % prompt
    \newcommand{\prompt}[4]{
        \llap{{\color{#2}[#3]: #4}}\vspace{-1.25em}
    }
    

    
    % Prevent overflowing lines due to hard-to-break entities
    \sloppy 
    % Setup hyperref package
    \hypersetup{
      breaklinks=true,  % so long urls are correctly broken across lines
      colorlinks=true,
      urlcolor=urlcolor,
      linkcolor=linkcolor,
      citecolor=citecolor,
      }
    % Slightly bigger margins than the latex defaults
    
    \geometry{verbose,tmargin=1in,bmargin=1in,lmargin=1in,rmargin=1in}
    
    
\author{Mats Hoem Olsen}
\date{19/2-2020}
    \begin{document}
    
    
    \maketitle
    
    

    
    \hypertarget{oppgave-1-mangkantkunst.py}{%
\section{Oppgave 1
(mangkantkunst.py)}\label{oppgave-1-mangkantkunst.py}}

Oppgaven sier vi skal lage et program som tegner et polygram.

    \begin{tcolorbox}[breakable, size=fbox, boxrule=1pt, pad at break*=1mm,colback=cellbackground, colframe=cellborder]
\prompt{In}{incolor}{1}{\hspace{4pt}}
\begin{Verbatim}[commandchars=\\\{\}]
\PY{k+kn}{import} \PY{n+nn}{turtle} \PY{c+c1}{\PYZsh{}! ALDRI importer en hel modul uten at du vet hva som er inni den, og jeg vet ikke...}

\PY{n}{user\PYZus{}input} \PY{o}{=} \PY{n+nb}{int}\PY{p}{(}\PY{n+nb}{input}\PY{p}{(}\PY{l+s+s2}{\PYZdq{}}\PY{l+s+s2}{Gi meg et tall større enn 0}\PY{l+s+se}{\PYZbs{}n}\PY{l+s+s2}{\PYZdq{}}\PY{p}{)}\PY{p}{)}
\PY{k}{if} \PY{n}{user\PYZus{}input} \PY{o}{\PYZlt{}}\PY{o}{=} \PY{l+m+mi}{0}\PY{p}{:}
	\PY{k}{raise} \PY{n+ne}{ValueError}\PY{p}{(}\PY{l+s+s2}{\PYZdq{}}\PY{l+s+s2}{input må være større enn 0}\PY{l+s+s2}{\PYZdq{}}\PY{p}{)}
\PY{n}{angle} \PY{o}{=} \PY{l+m+mi}{360}\PY{o}{/}\PY{n}{user\PYZus{}input}
\PY{n}{length} \PY{o}{=} \PY{l+m+mi}{80} \PY{c+c1}{\PYZsh{} 1 = 1 anus}

\PY{k}{for} \PY{n}{shit} \PY{o+ow}{in} \PY{n+nb}{range}\PY{p}{(}\PY{n}{user\PYZus{}input}\PY{p}{)}\PY{p}{:}
	\PY{n}{turtle}\PY{o}{.}\PY{n}{forward}\PY{p}{(}\PY{n}{length}\PY{p}{)}
	\PY{n}{turtle}\PY{o}{.}\PY{n}{right}\PY{p}{(}\PY{n}{angle}\PY{p}{)}

\PY{n}{turtle}\PY{o}{.}\PY{n}{done}\PY{p}{(}\PY{p}{)}

\PY{l+s+sd}{\PYZdq{}\PYZdq{}\PYZdq{}}
\PY{l+s+sd}{cmd: python.exe mangekantkunst.py}
\PY{l+s+sd}{Gi meg et tall større enn 0}
\PY{l+s+sd}{1}
\PY{l+s+sd}{*turtle starter*}
\PY{l+s+sd}{*tegner en linje*}
\PY{l+s+sd}{*program termineres*}
\PY{l+s+sd}{\PYZdq{}\PYZdq{}\PYZdq{}}
\end{Verbatim}
\end{tcolorbox}

    \begin{Verbatim}[commandchars=\\\{\}]
Gi meg et tall større enn 0
 4
\end{Verbatim}

    Dette ga oss en 4 siders sirkel (jeg tar nå en ekte sirkel til å ha
uendelig mange sider)

    \hypertarget{oppgave-2-mangkantkunst2.py}{%
\section{Oppgave 2
(mangkantkunst2.py)}\label{oppgave-2-mangkantkunst2.py}}

Oppgaven utvider Oppgave 1 ved å rotere figuren x ganger rundt sentrum.

    \begin{tcolorbox}[breakable, size=fbox, boxrule=1pt, pad at break*=1mm,colback=cellbackground, colframe=cellborder]
\prompt{In}{incolor}{1}{\hspace{4pt}}
\begin{Verbatim}[commandchars=\\\{\}]
\PY{k+kn}{import} \PY{n+nn}{turtle} \PY{c+c1}{\PYZsh{}! samme beskjed som i forrige oppgave...}

\PY{n}{user\PYZus{}input} \PY{o}{=} \PY{n+nb}{int}\PY{p}{(}\PY{n+nb}{input}\PY{p}{(}\PY{l+s+s2}{\PYZdq{}}\PY{l+s+s2}{gi et tall større enn 0}\PY{l+s+se}{\PYZbs{}n}\PY{l+s+s2}{\PYZdq{}}\PY{p}{)}\PY{p}{)}
\PY{k}{if} \PY{n}{user\PYZus{}input} \PY{o}{\PYZlt{}}\PY{o}{=} \PY{l+m+mi}{0}\PY{p}{:}
	\PY{k}{raise} \PY{n+ne}{ValueError}\PY{p}{(}\PY{l+s+s2}{\PYZdq{}}\PY{l+s+s2}{input må være større enn 0}\PY{l+s+s2}{\PYZdq{}}\PY{p}{)}
\PY{n}{blad\PYZus{}tall} \PY{o}{=} \PY{n+nb}{int}\PY{p}{(}\PY{n+nb}{input}\PY{p}{(}\PY{l+s+s2}{\PYZdq{}}\PY{l+s+s2}{hvor mange blader vil du ha? n \PYZgt{} 0}\PY{l+s+se}{\PYZbs{}n}\PY{l+s+s2}{\PYZdq{}}\PY{p}{)}\PY{p}{)}
\PY{k}{if} \PY{n}{blad\PYZus{}tall} \PY{o}{\PYZlt{}}\PY{o}{=} \PY{l+m+mi}{0}\PY{p}{:}
	\PY{k}{raise} \PY{n+ne}{ValueError}\PY{p}{(}\PY{l+s+s2}{\PYZdq{}}\PY{l+s+s2}{input må være større enn 0}\PY{l+s+s2}{\PYZdq{}}\PY{p}{)}

\PY{n}{length} \PY{o}{=} \PY{l+m+mi}{80}
\PY{n}{penta} \PY{o}{=} \PY{l+m+mi}{360}\PY{o}{/}\PY{n}{user\PYZus{}input} \PY{c+c1}{\PYZsh{}vi sier navnet er informativt... Dette er veldig kort kode, gled deg til oppgave 3}
\PY{n}{blad} \PY{o}{=} \PY{l+m+mi}{360}\PY{o}{/}\PY{n}{blad\PYZus{}tall}

\PY{k}{for} \PY{n}{b} \PY{o+ow}{in} \PY{n+nb}{range}\PY{p}{(}\PY{n}{blad\PYZus{}tall}\PY{p}{)}\PY{p}{:}
	\PY{k}{for} \PY{n}{l} \PY{o+ow}{in} \PY{n+nb}{range}\PY{p}{(}\PY{n}{user\PYZus{}input}\PY{p}{)}\PY{p}{:}
		\PY{n}{turtle}\PY{o}{.}\PY{n}{forward}\PY{p}{(}\PY{n}{length}\PY{p}{)}
		\PY{n}{turtle}\PY{o}{.}\PY{n}{right}\PY{p}{(}\PY{n}{penta}\PY{p}{)}
	\PY{n}{turtle}\PY{o}{.}\PY{n}{right}\PY{p}{(}\PY{n}{blad}\PY{p}{)}
\PY{n}{turtle}\PY{o}{.}\PY{n}{done}\PY{p}{(}\PY{p}{)}

\PY{l+s+sd}{\PYZdq{}\PYZdq{}\PYZdq{}}
\PY{l+s+sd}{cmd: python.exe mangekantkunst2.py}
\PY{l+s+sd}{gi et tall større enn 0}
\PY{l+s+sd}{6}
\PY{l+s+sd}{hvor mange blader vil du ha?}
\PY{l+s+sd}{3}
\PY{l+s+sd}{*turtle kjører*}
\PY{l+s+sd}{*turtle tegner 3 6\PYZhy{}kanter hvor sidene er rører hverandre.*}
\PY{l+s+sd}{cmd: python.exe mangekantkunst2.py}
\PY{l+s+sd}{gi et tall større enn 0}
\PY{l+s+sd}{0}
\PY{l+s+sd}{*viser en feilmelding*}
\PY{l+s+sd}{*Program termineres*}

\PY{l+s+sd}{\PYZdq{}\PYZdq{}\PYZdq{}}
\end{Verbatim}
\end{tcolorbox}

    \begin{Verbatim}[commandchars=\\\{\}]
gi et tall større enn 0
 4
hvor mange blader vil du ha? n > 0
 6
\end{Verbatim}

    \hypertarget{oppgave-3-hemmelig_beskjed.py}{%
\section{Oppgave 3
(hemmelig\_beskjed.py)}\label{oppgave-3-hemmelig_beskjed.py}}

For å være ærlig leste jeg ikke oppgaven nøye\ldots{} Så her får dere 25
tegn med egen compiler som lager runer. Les README.txt før dere kjører
programet.

    Vi starter med hva som skal skrives. Siden mitt program er automatiser
så spiller det ikke noen rolle hva vi skriver siden LEXEN ordner opp for
oss. Det er laget en kode som debugger output-en for retting.

    \begin{tcolorbox}[breakable, size=fbox, boxrule=1pt, pad at break*=1mm,colback=cellbackground, colframe=cellborder]
\prompt{In}{incolor}{ }{\hspace{4pt}}
\begin{Verbatim}[commandchars=\\\{\}]
\PY{n}{temp} \PY{o}{=} \PY{l+s+s2}{\PYZdq{}}\PY{l+s+s2}{\PYZdq{}}
\PY{k}{for} \PY{n}{le} \PY{o+ow}{in} \PY{n}{encodet}\PY{p}{:}
	\PY{n}{temp} \PY{o}{+}\PY{o}{=} \PY{n+nb}{chr}\PY{p}{(}\PY{n}{le}\PY{p}{)}
\PY{n+nb}{print}\PY{p}{(}\PY{n}{temp}\PY{p}{)}
\end{Verbatim}
\end{tcolorbox}

    Det lexen gir oss er unicode versjonen av bokstavene, men ikke tallene
siden vikinger ikke hadde tall, de hadde noen form for tall skriving men
for det meste skrev de ned tallene på ord form. Lexen min ordner opp i
dette for oss. ``temp += chr(le)'' tar disse tallverdiene og konverterer
dem til str som er ``lesbare'', gitt du har den nyeste versjonen av
utf-8. Dette blir brukt i selve loopen.

    \begin{tcolorbox}[breakable, size=fbox, boxrule=1pt, pad at break*=1mm,colback=cellbackground, colframe=cellborder]
\prompt{In}{incolor}{ }{\hspace{4pt}}
\begin{Verbatim}[commandchars=\\\{\}]
\PY{n}{turtle}\PY{o}{.}\PY{n}{penup}\PY{p}{(}\PY{p}{)} \PY{c+c1}{\PYZsh{}Må ha for å sette posisjon}
\PY{k}{for} \PY{n}{letter} \PY{o+ow}{in} \PY{n}{encodet}\PY{p}{:}
	\PY{n}{hight} \PY{o}{=} \PY{n}{hight} \PY{o}{\PYZhy{}} \PY{l+m+mi}{15} \PY{k}{if} \PY{n}{counter} \PY{o}{\PYZpc{}} \PY{n}{limit} \PY{o}{==} \PY{l+m+mi}{0} \PY{k}{else} \PY{n}{hight} \PY{c+c1}{\PYZsh{} Når man ønsker å gå lavere}
	\PY{n}{counter} \PY{o}{=} \PY{l+m+mi}{0} \PY{k}{if} \PY{n}{counter} \PY{o}{\PYZpc{}} \PY{n}{limit} \PY{o}{==} \PY{l+m+mi}{0} \PY{k}{else} \PY{n}{counter} \PY{c+c1}{\PYZsh{}Speeeeeeeeeed}
	\PY{n}{avdrag} \PY{o}{=} \PY{l+m+mi}{0} \PY{k}{if} \PY{n}{counter} \PY{o}{\PYZpc{}} \PY{n}{limit} \PY{o}{==} \PY{l+m+mi}{0} \PY{k}{else} \PY{n}{avdrag} \PY{c+c1}{\PYZsh{} ved å la avdraget være uforandret får vi skråtekst}
	\PY{n}{turtle}\PY{o}{.}\PY{n}{goto}\PY{p}{(}\PY{o}{\PYZhy{}}\PY{l+m+mi}{250} \PY{o}{+} \PY{l+m+mi}{10}\PY{o}{*}\PY{n}{counter} \PY{o}{+} \PY{l+m+mi}{1} \PY{o}{\PYZhy{}} \PY{n}{avdrag}\PY{p}{,}\PY{n}{screen\PYZus{}hight} \PY{o}{/} \PY{l+m+mi}{2} \PY{o}{+} \PY{n}{hight} \PY{o}{\PYZhy{}} \PY{l+m+mi}{25}\PY{p}{)} \PY{c+c1}{\PYZsh{}Start på font}
	\PY{n}{letterl} \PY{o}{=} \PY{n+nb}{chr}\PY{p}{(}\PY{n}{letter}\PY{p}{)} \PY{c+c1}{\PYZsh{}oversetter fra int til en bokstav, fjernes ved mangel på lexen}
	\PY{n}{turtle}\PY{o}{.}\PY{n}{pendown}\PY{p}{(}\PY{p}{)}
	\PY{n}{avdrag} \PY{o}{+}\PY{o}{=} \PY{n+nb}{eval}\PY{p}{(}\PY{l+s+s2}{\PYZdq{}}\PY{l+s+s2}{bm.tegn\PYZus{}}\PY{l+s+si}{\PYZob{}\PYZcb{}}\PY{l+s+s2}{()}\PY{l+s+s2}{\PYZdq{}}\PY{o}{.}\PY{n}{format}\PY{p}{(}\PY{n}{letterl}\PY{p}{)}\PY{p}{)} \PY{k}{if} \PY{n}{letterl} \PY{o+ow}{not} \PY{o+ow}{in} \PY{p}{[}\PY{l+s+s2}{\PYZdq{}}\PY{l+s+s2}{ }\PY{l+s+s2}{\PYZdq{}}\PY{p}{,}\PY{l+s+s2}{\PYZdq{}}\PY{l+s+s2}{.}\PY{l+s+s2}{\PYZdq{}}\PY{p}{,}\PY{l+s+s2}{\PYZdq{}}\PY{l+s+s2}{,}\PY{l+s+s2}{\PYZdq{}}\PY{p}{,}\PY{l+s+s2}{\PYZdq{}}\PY{l+s+s2}{᛬}\PY{l+s+s2}{\PYZdq{}}\PY{p}{]} \PY{k}{else} \PY{n+nb}{eval}\PY{p}{(}\PY{l+s+s2}{\PYZdq{}}\PY{l+s+s2}{bm.tegn\PYZus{}}\PY{l+s+si}{\PYZob{}\PYZcb{}}\PY{l+s+s2}{()}\PY{l+s+s2}{\PYZdq{}}\PY{o}{.}\PY{n}{format}\PY{p}{(}\PY{p}{\PYZob{}}\PY{l+s+s2}{\PYZdq{}}\PY{l+s+s2}{,}\PY{l+s+s2}{\PYZdq{}}\PY{p}{:}\PY{l+s+s2}{\PYZdq{}}\PY{l+s+s2}{kom}\PY{l+s+s2}{\PYZdq{}}\PY{p}{,}\PY{l+s+s2}{\PYZdq{}}\PY{l+s+s2}{.}\PY{l+s+s2}{\PYZdq{}}\PY{p}{:}\PY{l+s+s2}{\PYZdq{}}\PY{l+s+s2}{punk}\PY{l+s+s2}{\PYZdq{}}\PY{p}{,}\PY{l+s+s2}{\PYZdq{}}\PY{l+s+s2}{ }\PY{l+s+s2}{\PYZdq{}}\PY{p}{:}\PY{l+s+s2}{\PYZdq{}}\PY{l+s+s2}{sp}\PY{l+s+s2}{\PYZdq{}}\PY{p}{,}\PY{l+s+s2}{\PYZdq{}}\PY{l+s+s2}{᛬}\PY{l+s+s2}{\PYZdq{}}\PY{p}{:}\PY{l+s+s2}{\PYZdq{}}\PY{l+s+s2}{sp}\PY{l+s+s2}{\PYZdq{}}\PY{p}{\PYZcb{}}\PY{p}{[}\PY{n}{letterl}\PY{p}{]}\PY{p}{)}\PY{p}{)} \PY{c+c1}{\PYZsh{}tegner og opdaterer avdrag, SPEED CODE!}
	\PY{n}{turtle}\PY{o}{.}\PY{n}{setheading}\PY{p}{(}\PY{l+m+mi}{0}\PY{p}{)}
	\PY{n}{counter} \PY{o}{+}\PY{o}{=} \PY{l+m+mi}{1} \PY{c+c1}{\PYZsh{}Oppdatering}
\PY{n}{turtle}\PY{o}{.}\PY{n}{hideturtle}\PY{p}{(}\PY{p}{)} \PY{c+c1}{\PYZsh{}trengs ikke men bør være med for å lese siste font.}
\PY{n}{turtle}\PY{o}{.}\PY{n}{done}\PY{p}{(}\PY{p}{)}
\end{Verbatim}
\end{tcolorbox}

    Variablene hight, counter, og avdrag holder programmet sammen og
organisert.
\begin{enumerate}
\item hight; holder styr på hvor høyt opp turtle skal tegen. Denne blir forandret jo lengere inn i beskjeden vi kommer. 
\item counter;teller hvilke font vi er på og dermed oppdatere alle andre variabler. 
\item
avdrag; hver font er ikke like store og vil derfor være mindre eller
større. Dette reflekteres i linje 51.
\end{enumerate}
 Istedet for å hente avdraget til
hver font for så tegne valgte jeg heller at funksjonene returnerer
avdraget etter de har tegnet fonten. Dette gjør at koden bli mindre og
mer automatisert.

    Her er et eksempel på en tegn modul:

    \begin{tcolorbox}[breakable, size=fbox, boxrule=1pt, pad at break*=1mm,colback=cellbackground, colframe=cellborder]
\prompt{In}{incolor}{ }{\hspace{4pt}}
\begin{Verbatim}[commandchars=\\\{\}]
\PY{k}{def} \PY{n+nf}{tegn\PYZus{}ᚱ}\PY{p}{(}\PY{p}{)}\PY{p}{:}
	\PY{n}{turtle}\PY{o}{.}\PY{n}{left}\PY{p}{(}\PY{l+m+mi}{90}\PY{p}{)}
	\PY{n}{turtle}\PY{o}{.}\PY{n}{forward}\PY{p}{(}\PY{l+m+mi}{8}\PY{p}{)}
	\PY{n}{turtle}\PY{o}{.}\PY{n}{right}\PY{p}{(}\PY{l+m+mi}{90}\PY{p}{)}
	\PY{n}{turtle}\PY{o}{.}\PY{n}{forward}\PY{p}{(}\PY{l+m+mi}{1}\PY{p}{)}
	\PY{n}{turtle}\PY{o}{.}\PY{n}{right}\PY{p}{(}\PY{l+m+mi}{45}\PY{p}{)}
	\PY{n}{turtle}\PY{o}{.}\PY{n}{forward}\PY{p}{(}\PY{l+m+mi}{3}\PY{p}{)}
	\PY{n}{turtle}\PY{o}{.}\PY{n}{right}\PY{p}{(}\PY{l+m+mi}{90}\PY{p}{)}
	\PY{n}{turtle}\PY{o}{.}\PY{n}{forward}\PY{p}{(}\PY{l+m+mi}{3}\PY{p}{)}
	\PY{n}{turtle}\PY{o}{.}\PY{n}{left}\PY{p}{(}\PY{l+m+mi}{90}\PY{p}{)}
	\PY{n}{turtle}\PY{o}{.}\PY{n}{forward}\PY{p}{(}\PY{l+m+mi}{7}\PY{p}{)}
	\PY{n}{turtle}\PY{o}{.}\PY{n}{penup}\PY{p}{(}\PY{p}{)}
	\PY{k}{return} \PY{l+m+mi}{2} \PY{c+c1}{\PYZsh{}avdrag}
\end{Verbatim}
\end{tcolorbox}

    Som en kan se er meste parten av funksjonen et kall på turtle til å
tegne utifra der den er. Return verdien beskriver hvor mye som er igjen
av fonten. Siden mesteparten av tegnene vil være mindre eller lik 8 px
brei vil verdiene for manglende plass være positiv mens ekstra bruk av
plass vil være negativ.

    \hypertarget{tillegg}{%
\section{Tillegg}\label{tillegg}}

Lex.py analyserer text som den tar inn med hensyn til ordboken som vi
mater den. Den deler teksten opp inn i ord forså til bokstaver og tall.
Tall blir håndtert av en ennen klasse som vil dekomponere tallet til
bokstaver, men den kjører ikke 20- tallsystem som vi er vant med, den
kjører 10-tallsystem. Dette vil si at istedet for tjue gir den oss ``to
ti'' som er gramatisk riktig og logisk tiktig siden vi har 2 av 10
istedet for 1 av 20. Det andre systemet som kjører i ``lex'' er at den
vil først se etter bokstaver som ikke støttes av ordboken vi ga den for
så bytte de ut med relevante forslag også gitt av ordboken. Etter
konverteringen vil den gå igjenom blokker av ordene for å finne
``combo''-er, disse er spesielle i den form at selv om de kan erstasses
bokstav for bokstav har vi ekvivalente tegn som vil minke tekst lengde.
Det siste den vil gjøre er å konvertere bokstaver med relevante
bokstaver i ordboken. Det vi får til slutt er en liste med int som
representerer hvilke unicode bokstav det er. Mellomrom vil også
konverteres i dette tilfelle siden vikinger ikke hadde vanlige
mellomrom. Andre spesial tegn vil ikke erstates men istedet gjort om til
int som resten i riktig unicode. Aksagner på bokstaver vil fjernes med
modulen ``unidecode'' siden den vil gjøre arbeidet litt mer vanskelig,
med mindre vi kan simpelten legge til aksagn på runer i dette tilfellet.


    % Add a bibliography block to the postdoc
    
    
    
    \end{document}
